\documentclass[spanish,a4paper,11pt,twoside]{report}

%%%%%%%%%%%%%%%%%%%%%%%%%%%%%%%%%%%%%%%%%%%%%%%%%%%%%%%%%%%%%%%%%%%%%%%%%%%%%%%
\usepackage[dvips]{graphicx}
\usepackage[dvips]{epsfig}
\usepackage[latin1]{inputenc}
\usepackage[spanish]{babel}
\usepackage{alltt}
\usepackage{templates/algorithm}
\usepackage{templates/algorithmic}
\usepackage{templates/multirow}
\usepackage{amsmath}
%%%%%%%%%%%%%%%%%%%%%%%%%%%%%%%%%%%%%%%%%%%%%%%%%%%%%%%%%%%%%%%%%%%%%%%%%%%%%%%

\newcommand{\SONY}{{\sc Sony}}
\newcommand{\MICROSOFT}{{\sc Microsoft}}
\newcommand{\GCC}{\textsf{\textsc{G}CC}}
\newcommand{\INTEL}{\textsf{\textsc{I}ntel}}

%%% Traducimos el pseudocodigo
\renewcommand{\algorithmicwhile}{\textbf{mientras}}
\renewcommand{\algorithmicend}{\textbf{fin}}
\renewcommand{\algorithmicdo}{\textbf{hacer}}
\renewcommand{\algorithmicif}{\textbf{si}}
\renewcommand{\algorithmicthen}{\textbf{entonces}}
\renewcommand{\algorithmicrepeat}{\textbf{repetir}}
\renewcommand{\algorithmicuntil}{\textbf{hasta que}}
\renewcommand{\algorithmicelse}{\textbf{en otro caso}}
\renewcommand{\algorithmicfor}{\textbf{para}}

%\newcommand{\RETURN}{\textbf{retornar} }
\newcommand{\RET}{\STATE \textbf{retornar} }
\newcommand{\TO}{\textbf{hasta} }
\newcommand{\AND}{\textbf{y} }
\newcommand{\OR}{\textbf{o} }

%%%%%%%%%%%%%%%%% Creamos un entorno para listar c�digo fuente %%%%%%%%%%%%%%%
\newenvironment{sourcecode}
{\begin{list}{}{\setlength{\leftmargin}{1em}}\item\scriptsize\bfseries}
{\end{list}}

\newenvironment{littlesourcecode}
{\begin{list}{}{\setlength{\leftmargin}{1em}}\item\tiny\bfseries}
{\end{list}}

\newenvironment{summary}
{\par\noindent\begin{center}\textbf{Abstract}\end{center}\begin{itshape}\par\noindent}
{\end{itshape}}

\newenvironment{keywords}
{\begin{list}{}{\setlength{\leftmargin}{1em}}\item[\hskip\labelsep \bfseries Keywords:]}
{\end{list}}

\newenvironment{palabrasClave}
{\begin{list}{}{\setlength{\leftmargin}{1em}}\item[\hskip\labelsep \bfseries Palabras clave:]}
{\end{list}}


%%%%%%%%%%%%%%%%%%%%%%%%%%%%%%%%%%%%%%%%%%%%%%%%%%%%%%%%%%%%%%%%%%%%%%%%%%%%%%%
% Format
%%%%%%%%%%%%%%%%%%%%%%%%%%%%%%%%%%%%%%%%%%%%%%%%%%%%%%%%%%%%%%%%%%%%%%%%%%%%%%%

%%\topmargin -4 mm
%\topmargin -21 mm
%\headheight 10 mm
%\headsep 10 mm

%\textheight 229 mm
%\textheight 246 mm

%\oddsidemargin -5.4 mm
%\evensidemargin -5.4 mm
\oddsidemargin 5 mm
\evensidemargin 5 mm

%\oddsidemargin -3 mm
%\evensidemargin -3 mm

%\textwidth 17 cm
\textwidth 15 cm
%\columnsep 10 mm

\input{amssym.def}

%%%%%%%%%%%%%%%%%%%%%%%%%%%%%%%%%%%%%%%%%%%%%%%%%%%%%%%%%%%%%%%%%%%%%%%%%%%%%%%

\begin{document}

%%%%%%%%%%%%%%%%%%%%%%%%%%%%%%%%%%%%%%%%%%%%%%%%%%%%%%%%%%%%%%%%%%%%%%%%%%%%%%%
% First Page 
%%%%%%%%%%%%%%%%%%%%%%%%%%%%%%%%%%%%%%%%%%%%%%%%%%%%%%%%%%%%%%%%%%%%%%%%%%%%%%%
\pagestyle{empty}
\thispagestyle{empty}


\newcommand{\HRule}{\rule{\linewidth}{1mm}}
\setlength{\parindent}{0mm}
\setlength{\parskip}{0mm}
\vspace*{\stretch{1}}

\begin{center}
\includegraphics[width=0.2\textwidth]{images/logotipo-secundario-ULL}\\[0.25cm]
\end{center}

\HRule
\begin{center}
        {\Huge M�todo de Simpson} \\[2.5mm] 
        {\Huge Comparaci�n de f�rmulas del m�todo} \\[2.5mm]
        {\Large Nadia Chinea,Tania Guti�rrez, Melanie Hern�ndez} \\[5mm]
        {\Large \textit{Grupo($2\mid H$) }} \\[5mm]


        {\em T�cnicas Experimentales. $1^{er}$ curso. $2^{do}$ semestre} \\[5mm]
        Lenguajes y Sistemas Inform�ticos \\[5mm]
        Facultad de Matem�ticas \\[5mm]
        
        Universidad de La Laguna \\
\end{center}
\HRule
\vspace*{\stretch{2}}
\begin{center}
  La Laguna, \today 
\end{center}
%%%%%%%%%%%%%%%%%%%%%%%%%%%%%%%%%%%%%%%%%%%%%%%%%%%%%%%%%%%%%%%%%%%%%%%%%%%%%%%

%%%%%%%%%%%%%%%%%%%%%%%%%%%%%%%%%%%%%%%%%%%%%%%%%%%%%%%%%%%%%%%%%%%%%%%%%%%%%%%
\newpage{\pagestyle{empty}\cleardoublepage}

\pagestyle{myheadings} %my head defined by markboth or markright
% No funciona bien \markboth sin "twoside" en \documentclass, pero al
% ponerlo se dan un mont�n de errores de underfull \vbox, con lo que no se
% ha puesto.
\markboth{N.Chinea, T.Guti�rrez, M.Hern�ndez}{M�todo de Simpson}

%%%%%%%%%%%%%%%%%%%%%%%%%%%%%%%%%%%%%%%%%%%%%%%%%%%%%%%%%%%%%%%%%%%%%%%%%%%%%%%
%Numeracion en romanos
\renewcommand{\thepage}{\roman{page}}
\setcounter{page}{1}

%%%%%%%%%%%%%%%%%%%%%%%%%%%%%%%%%%%%%%%%%%%%%%%%%%%%%%%%%%%%%%%%%%%%%%%%%%%%%%%

\tableofcontents

%%%%%%%%%%%%%%%%%%%%%%%%%%%%%%%%%%%%%%%%%%%%%%%%%%%%%%%%%%%%%%%%%%%%%%%%%%%%%%%
\newpage{\pagestyle{empty}\cleardoublepage}

\listoffigures

%%%%%%%%%%%%%%%%%%%%%%%%%%%%%%%%%%%%%%%%%%%%%%%%%%%%%%%%%%%%%%%%%%%%%%%%%%%%%%%
\newpage{\pagestyle{empty}\cleardoublepage}

\listoftables

%%%%%%%%%%%%%%%%%%%%%%%%%%%%%%%%%%%%%%%%%%%%%%%%%%%%%%%%%%%%%%%%%%%%%%%%%%%%%%%
\newpage{\pagestyle{empty}\cleardoublepage}

%%%%%%%%%%%%%%%%%%%%%%%%%%%%%%%%%%%%%%%%%%%%%%%%%%%%%%%%%%%%%%%%%%%%%%%%%%%%%%%
%Numeracion a partir del capitulo I
\renewcommand{\thepage}{\arabic{page}}
\setcounter{page}{1}

\setlength{\parindent}{5mm}

%%%%%%%%%%%%%%%%%%%%%%%%%%%%%%%%%%%%%%%%%%%%%%%%%%%%%%%%%%%%%%%%%%%%%%%%%%%%%%%
\chapter{Motivaci�n y objetivos}
\label{chapter:obj}

%%%%%%%%%%%%%%%%%%%%%%%%%%%%%%%%%%%%%%%%%%%%%%%%%%%%%%%%%%%%%%%%%%%%%%%%%%%%%
% Chapter 1: Motivaci�n y Objetivos 
%%%%%%%%%%%%%%%%%%%%%%%%%%%%%%%%%%%%%%%%%%%%%%%%%%%%%%%%%%%%%%%%%%%%%%%%%%%%%%%


Para comenzar este trabajo, expondremos a continuaci�n los fines de la realizaci�n de este proyecto
que implica la resoluci�n de un problema con la utilizaci�n de un lenguaje de programaci�n interpretado.  

%---------------------------------------------------------------------------------
  \section{Objetivo principal}
\label{1:sec:1}
  La intenci�n principal para el desarrollo de este informe es el planteamiento de un experimento cuyo fin 
es el c�lculo del �rea de una funci�n dada de la forma m�s precisa posible. 
En este desarrollo interviene un m�todo de integraci�n num�rica, lo que conlleva no s�lo un aprendizaje en el �mbito inform�tico 
(referido a la utilizaci�n de algoritmos y sentencias l�gicas) sino que adem�s permite hacer un enfoque matem�tico 
que obliga a alcanzar una mayor destreza en �nalisis matem�tico\footnote{Es una rama de la ciencia matem�tica que se empieza a desarrollar a partir
 del inicio de la formulaci�n del c�lculo y estudia conceptos como la continuidad, la integraci�n y la diferenciabilidad de diversas formas}.

%---------------------------------------------------------------------------------
\section{Secci�n Dos}
\label{1:sec:2}
  Primer p�rrafo de la segunda secci�n.

\begin{itemize}
  \item Item 1
  \item Item 2
  \item Item 3
\end{itemize}



%%%%%%%%%%%%%%%%%%%%%%%%%%%%%%%%%%%%%%%%%%%%%%%%%%%%%%%%%%%%%%%%%%%%%%%%%%%%%%%
\chapter{Fundamentos te�ricos}
\label{chapter:teo}

%%%%%%%%%%%%%%%%%%%%%%%%%%%%%%%%%%%%%%%%%%%%%%%%%%%%%%%%%%%%%%%%%%%%%%%%%%%%%%%
% Chapter 2: Fundamentos Te�ricos 
%%%%%%%%%%%%%%%%%%%%%%%%%%%%%%%%%%%%%%%%%%%%%%%%%%%%%%%%%%%%%%%%%%%%%%%%%%%%%%%

%++++++++++++++++++++++++++++++++++++++++++++++++++++++++++++++++++++++++++++++

En este cap�tulo se han de presentar los antecedentes te�ricos y pr�cticos que
apoyan el tema objeto de la investigaci�n.

%++++++++++++++++++++++++++++++++++++++++++++++++++++++++++++++++++++++++++++++

\section{Ideas principales}
\label{2:sec:1}
Es impresindible que comencemos con la forma m�s com�n a la hora de hallar �reas de funciones, �sta es la integral definida.
La integraci�n es un concepto fundamental del c�lculo y del an�lisis matem�tico. B�sicamente, la integral es una generalizaci�n 
de la suma de infinitos sumandos,infinitamente peque�os. Cuando realizamos la integral de una funci�n estamos representando
el �rea limitada por la gr�fica de la funci�n, con signo positivo cuando la funci�n toma valores positivos y con signo negativo cuando toma valores menore que cero.

El problema surge cuando se intentan calcular �reas cuya integral no tiene resoluci�n con m�todos inmediatos ni con peque�as modificaciones que convierten 
una integral compleja en una inmedita, y es aqu� donde entra la larga lista de m�todos num�ricos, como por ejemplo el m�todo de de Taylor,Simpson, punto medio, 
regla del rect�ngulo,etc. En este caso, utilizaremos uno de los m�todos citados anteriormente, el m�todo de Simpson\footnote{En an�lisis num�rico, la regla o 
m�todo de Simpson(nombrada as� en honor de Thomas Simpson) y a veces llamada regla de Kepler, es un m�todo de integraci�n num�rica que se utiliza 
para obtener la aproximaci�n de una integral definida}

\section{Segundo apartado del segundo cap�tulo}
\label{2:sec:2}
Como ya sabemos, debemos establecer la funci�n con la que se realizar� todo el proyecto, �sta va a ser la siguiente:
\[ F(x)=\frac{1}{\sqrt{2\pi}} \text{e}^{\frac{-x^2}{2}}\]\par
Cuya integral no tiene soluci�n:
\[\int_{-1}^{1} \frac{1}{\sqrt{2\pi}}\text{e}^{\frac{-x^2}{2}} \text{d}x\] 
Las integrales definidas tienen la caracter�stica necesaria e impresindible de necesitar una acotaci�n en un intervalo. Obviamente, con la  regla de Simpson
tambi�n se debe especificar un intervalo, y es �ste precisamente, el que determina que f�rmula de Simpson es m�s apropiada. Antes que nada,debemos decir que
existen dos f�rmulas para nuestra regla, cuya elecci�n recae en el tama�o de la acotaci�n, es decir, para un intervalo peque�o, como por ejemplo [-1,1],
se debe utilizar la f�rmula simple del m�todo y para una cota superior, como lo es [2,23] es m�s correcto el uso de la f�rmula compuesta de Simpson.
La f�rmula simple se define como:
\[\int_{a}^{b} f(x)\text{d}x ~ \frac{b-a}{6}\big[f(a)+4f(m)+f(b)\big]\]\par
 
Por otra parte, la f�rmula compuesta aparece como:

\[\int_{a}^{b} f(x)\text{d}x ~ \frac{h}{3}\big[f(x_0)+2\sum_{j=1}{\frac{n}{2-1}} f(x_{2j})+4 \sum_{j=1}{\frac{n}{2}}f(x_{2j-1})+f(x_n)\]


En nuestro proyecto, se nos ha propuesto utilizar principalmente el intervalo [-1,1] pero debido al tipo de experimento que deseamos 
realizar, adem�s de �ste, otros intervalos de diferentes tama�os para que nos produzcan resultados variados y se determine, como se ha expuesto aqu�,
la considerable diferencia entre la utilizaci�n de cada una de las f�rmulas y la variaci�n entre las diferentes cotas.




%%%%%%%%%%%%%%%%%%%%%%%%%%%%%%%%%%%%%%%%%%%%%%%%%%%%%%%%%%%%%%%%%%%%%%%%%%%%%%%
\chapter{Procedimiento experimental}
\label{chapter:exp}

\input{tex/cap3.tex}

%%%%%%%%%%%%%%%%%%%%%%%%%%%%%%%%%%%%%%%%%%%%%%%%%%%%%%%%%%%%%%%%%%%%%%%%%%%%%%%
\chapter{Conclusiones}
\label{chapter:conclusiones}

%%%%%%%%%%%%%%%%%%%%%%%%%%%%%%%%%%%%%%%%%%%%%%%%%%%%%%%%%%%%%%%%%%%%%%%%%%%%%
% Chapter 4: Conclusiones y Trabajos Futuros 
%%%%%%%%%%%%%%%%%%%%%%%%%%%%%%%%%%%%%%%%%%%%%%%%%%%%%%%%%%%%%%%%%%%%%%%%%%%%%%%
%No compila al escribir algunas de las tíldes
En este trabajo se presento el desarrollo de la resolucion del area de una funcion por el metodo de Simpson.
El objetivo ha sido la confeccion de un informe redactado en codigo ~\LaTeX, y unas transparecias en Beamer con este mismo lenguaje. 
Cabe decir, que el lenguaje de programacion con el que se ha confeccionado el experimento ha sido el python.
Se podría decir que la primera conclusion que hemos obtenido ha sido la verificacion de que los intervalos menores se usan para la formula simple y 
que los mayores para la formula compuesta.
Tambien, hemos visto que en los programas en los que intervienen definiciones de funciones no muy complejas el tiempo de la CPU es minimo.
A parte de esto, se puede concluir diciendo que hemos obtenido facilidades a la hora de creacion de codigos para resolucion de problemas, ademas de haber 
logrado aumentar las capacidades que ya se tenian con ~\LaTeX

%%%%%%%%%%%%%%%%%%%%%%%%%%%%%%%%%%%%%%%%%%%%%%%%%%%%%%%%%%%%%%%%%%%%%%%%%%%%%%%

%%%%%%%%%%%%%%%%%%%%%%%%%%%%%%%%%%%%%%%%%%%%%%%%%%%%%%%%%%%%%%%%%%%%%%%%%%%%%%%
\newpage{\pagestyle{empty}\cleardoublepage}
\thispagestyle{empty}
\begin{appendix}

\chapter{Algoritmo de m�dulos de las f�rmulas y tiempo}
\label{appendix:1}

\input{tex/apendice1.tex}

\chapter{Algoritmo de Gr�ficas}
\label{appendix:2}

\input{tex/apendice2.tex}

\end{appendix}

%%%%%%%%%%%%%%%%%%%%%%%%%%%%%%%%%%%%%%%%%%%%%%%%%%%%%%%%%%%%%%%%%%%%%%%%%%%%%%%
\begin{thebibliography}{10}
 \bibitem {uno}Edicion de documentos en ~\LaTeX Andre Deprit, Antonio Felipe, Sebastian Ferrer 
 
 \bibitem{grifhing} URL: Wikipedia {M�todo de Simpson}
 Wikipedia.{\small   $ http: //es.wikipedia.org/wiki/Regla_de_Simpson$}
 
 \bibitem{python} Python / Jim Knowlton (2008)
 
 \bibitem{geo1} Calculo y Geometria analitica. Volumen1 / Sherman~.k~Stein, Anthony Barcellos
 
 \bibitem{geo2} Calculo con geometria analitica / Dennis ~G. Zill
 
\end{thebibliography}

%%%%%%%%%%%%%%%%%%%%%%%%%%%%%%%%%%%%%%%%%%%%%%%%%%%%%%%%%%%%%%%%%%%%%%%%%%%%%%%

\end{document}
