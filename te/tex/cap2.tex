%%%%%%%%%%%%%%%%%%%%%%%%%%%%%%%%%%%%%%%%%%%%%%%%%%%%%%%%%%%%%%%%%%%%%%%%%%%%%%%
% Chapter 2: Fundamentos Te�ricos 
%%%%%%%%%%%%%%%%%%%%%%%%%%%%%%%%%%%%%%%%%%%%%%%%%%%%%%%%%%%%%%%%%%%%%%%%%%%%%%%

%++++++++++++++++++++++++++++++++++++++++++++++++++++++++++++++++++++++++++++++

En este cap�tulo se han de presentar los antecedentes te�ricos y pr�cticos que
apoyan el tema objeto de la investigaci�n.

%++++++++++++++++++++++++++++++++++++++++++++++++++++++++++++++++++++++++++++++

\section{Primer apartado del segundo cap�tulo}
\label{2:sec:1}
  Primer p�rrafo de la primera secci�n.

\section{Segundo apartado del segundo cap�tulo}
\label{2:sec:2}
La f�rmula simple se define como:
\[\int_{a}^{b} f(x)\text{d}x ~ \frac{b-a}{6}\big[f(a)+4f(m)+f(b)\big]\]\par
 
Por otra parte, la f�rmula compuesta aparece como:

\[\int_{a}^{b} f(x)\text{d}x ~ \frac{h}{3}\big[f(x_0)+2\sum_{j=1}{\frac{n}{2-1}} f(x_{2j})+4 \sum_{j=1}{\frac{n}{2}}f(x_{2j-1})+f(x_n)\]


En nuestro proyecto, se nos ha propuesto utilizar principalmente el intervalo [-1,1] pero debido al tipo de experimento que deseamos 
realizar, adem�s de �ste, otros intervalos de diferentes tama�os para que nos produzcan resultados variados y se determine, como se ha expuesto aqu�,
la considerable diferencia entre la utilizaci�n de cada una de las f�rmulas y la variaci�n entre las diferentes cotas.


