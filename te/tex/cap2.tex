%%%%%%%%%%%%%%%%%%%%%%%%%%%%%%%%%%%%%%%%%%%%%%%%%%%%%%%%%%%%%%%%%%%%%%%%%%%%%%%
% Chapter 2: Fundamentos Te�ricos 
%%%%%%%%%%%%%%%%%%%%%%%%%%%%%%%%%%%%%%%%%%%%%%%%%%%%%%%%%%%%%%%%%%%%%%%%%%%%%%%

%++++++++++++++++++++++++++++++++++++++++++++++++++++++++++++++++++++++++++++++

En este cap�tulo se han de presentar los antecedentes te�ricos y pr�cticos que
apoyan el tema objeto de la investigaci�n.

%++++++++++++++++++++++++++++++++++++++++++++++++++++++++++++++++++++++++++++++

\section{Primer apartado del segundo cap�tulo}
\label{2:sec:1}
  Primer p�rrafo de la primera secci�n.

\section{Segundo apartado del segundo cap�tulo}
\label{2:sec:2}
Como ya sabemos, debemos establecer la funci�n con la que se realizar� todo el proyecto, �sta va a ser la siguiente:
\[ F(x)=\frac{1}{\sqrt{2\pi}} \text{e}^{\frac{-x^2}{2}}\]\par
Cuya integral no tiene soluci�n:
\[\int_{-1}^{1} \frac{1}{\sqrt{2\pi}}\text{e}^{\frac{-x^2}{2}} \text{d}x\] 
Las integrales definidas tienen la caracter�stica necesaria e impresindible de necesitar una acotaci�n en un intervalo. Obviamente, con la  regla de Simpson
tambi�n se debe especificar un intervalo, y es �ste precisamente, el que determina que f�rmula de Simpson es m�s apropiada. Antes que nada,debemos decir que
existen dos f�rmulas para nuestra regla, cuya elecci�n recae en el tama�o de la acotaci�n, es decir, para un intervalo peque�o, como por ejemplo [-1,1],
se debe utilizar la f�rmula simple del m�todo y para una cota superior, como lo es [2,23] es m�s correcto el uso de la f�rmula compuesta de Simpson.
La f�rmula simple se define como:
\[\int_{a}^{b} f(x)\text{d}x ~ \frac{b-a}{6}\big[f(a)+4f(m)+f(b)\big]\]\par
 
Por otra parte, la f�rmula compuesta aparece como:

\[\int_{a}^{b} f(x)\text{d}x ~ \frac{h}{3}\big[f(x_0)+2\sum_{j=1}{\frac{n}{2-1}} f(x_{2j})+4 \sum_{j=1}{\frac{n}{2}}f(x_{2j-1})+f(x_n)\]


En nuestro proyecto, se nos ha propuesto utilizar principalmente el intervalo [-1,1] pero debido al tipo de experimento que deseamos 
realizar, adem�s de �ste, otros intervalos de diferentes tama�os para que nos produzcan resultados variados y se determine, como se ha expuesto aqu�,
la considerable diferencia entre la utilizaci�n de cada una de las f�rmulas y la variaci�n entre las diferentes cotas.


