%%%%%%%%%%%%%%%%%%%%%%%%%%%%%%%%%%%%%%%%%%%%%%%%%%%%%%%%%%%%%%%%%%%%%%%%%%%%%
% Chapter 1: Motivaci�n y Objetivos 
%%%%%%%%%%%%%%%%%%%%%%%%%%%%%%%%%%%%%%%%%%%%%%%%%%%%%%%%%%%%%%%%%%%%%%%%%%%%%%%


Para comenzar este trabajo, expondremos a continuaci�n los fines de la realizaci�n de este proyecto
que implica la resoluci�n de un problema con la utilizaci�n de un lenguaje de programaci�n interpretado.  

%---------------------------------------------------------------------------------
  \section{Objetivo principal}
\label{1:sec:1}
  La intenci�n principal para el desarrollo de este informe es el planteamiento de un experimento cuyo fin 
es el c�lculo del �rea de una funci�n dada de la forma m�s precisa posible. 
En este desarrollo interviene un m�todo de integraci�n num�rica, lo que conlleva no s�lo un aprendizaje en el �mbito inform�tico 
(referido a la utilizaci�n de algoritmos y sentencias l�gicas) sino que adem�s permite hacer un enfoque matem�tico 
que obliga a alcanzar una mayor destreza en �nalisis matem�tico\footnote{Es una rama de la ciencia matem�tica que se empieza a desarrollar a partir
 del inicio de la formulaci�n del c�lculo y estudia conceptos como la continuidad, la integraci�n y la diferenciabilidad de diversas formas}.

%---------------------------------------------------------------------------------
\section{Secci�n Dos}
\label{1:sec:2}
  Primer p�rrafo de la segunda secci�n.

\begin{itemize}
  \item Item 1
  \item Item 2
  \item Item 3
\end{itemize}

