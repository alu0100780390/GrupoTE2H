\section{Otro apendice: Seccion 1}
\label{Apendice2:label}






\section{Grafica de intervalo [-1,4]}
\label{sec7}
\begin{center}
\begin{footnotesize}
\begin{verbatim}
   
   Nombre del fichero:
    tiempo_grafica.2.8.py

   Autoras de dicho fichero:
    Nadia Chinea Chinea 
    Tania Gutierrez Gutierrez 
    Melanie Hernandez Alonso

   #!/usr/bin/python
   from matplotlib.pylab import *

   def s(t):
     return 0.0539969133913

   def c(t):
     return 0.0278711396576 

   t = linspace(2, 8, 51)  # 51 puntos entre 2 y 8
   y = zeros(len(t))       # reserva memoria para y con elementos flotantes
   for i in xrange(len(t)):
     y[i] = s(t[i])

   tc = linspace(2, 8, 51)  # 51 puntos entre 2 y 8
   yc = zeros(len(t))       # reserva memoria para y con elementos flotantes
   for i in xrange(len(t)):
     yc[i] = c(tc[i])

   plot(t,y)
   plot(tc,yc)
   xlim(0.0, 10.0)
   ylim(0.01, 0.07)
   savefig('3intervalo.eps')
   show()  
   
\end{verbatim}
\end{footnotesize}
\end{center}




\begin{center}
\begin{footnotesize}

\begin{verbatim}
Texto
\end{verbatim}

\end{footnotesize}
\end{center}

\section{Otro apendice: Seccion 2}
\label{Apendice2:label2}

\begin{center}
\begin{footnotesize}

\begin{verbatim}
Texto
\end{verbatim}


\end{footnotesize}
\end{center}
