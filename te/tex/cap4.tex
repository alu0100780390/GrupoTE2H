%%%%%%%%%%%%%%%%%%%%%%%%%%%%%%%%%%%%%%%%%%%%%%%%%%%%%%%%%%%%%%%%%%%%%%%%%%%%%
% Chapter 4: Conclusiones y Trabajos Futuros 
%%%%%%%%%%%%%%%%%%%%%%%%%%%%%%%%%%%%%%%%%%%%%%%%%%%%%%%%%%%%%%%%%%%%%%%%%%%%%%%
%No compila al escribir algunas de las tíldes
En este trabajo se presento el desarrollo de la resolucion del area de una funcion por el metodo de Simpson.
El objetivo ha sido la confeccion de un informe redactado en codigo ~\LaTeX, y unas transparecias en Beamer con este mismo lenguaje. 
Cabe decir, que el lenguaje de programacion con el que se ha confeccionado el experimento ha sido el python.
Se podría decir que la primera conclusion que hemos obtenido ha sido la verificacion de que los intervalos menores se usan para la formula simple y 
que los mayores para la formula compuesta.
Tambien, hemos visto que en los programas en los que intervienen definiciones de funciones no muy complejas el tiempo de la CPU es minimo.
A parte de esto, se puede concluir diciendo que hemos obtenido facilidades a la hora de creacion de codigos para resolucion de problemas, ademas de haber 
logrado aumentar las capacidades que ya se tenian con ~\LaTeX