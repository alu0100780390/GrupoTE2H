\section{Algoritmo XXX}
\label{Apendice1:XXX}

\begin{center}
\begin{footnotesize}
\begin{verbatim}
###################################################################################
# Fichero .py
###################################################################################
#
# AUTORES
#   
# FECHA
#
# DESCRIPCION
#
###################################################################################
\end{verbatim}
\end{footnotesize}
\end{center}

\section{Algoritmo YYY}
\label{Apendice1:YYY}

\begin{center}
\begin{footnotesize}
\begin{verbatim}
/###################################################################################
 # Fichero .h
 ###################################################################################
 #
 # AUTORES
 #
 # FECHA
 #
 # DESCRIPCION
 #
 ##################################################################################
\end{verbatim}
\end{footnotesize}
\end{center}

      
\section{Algoritmo de tiempo de ejecucion, compuesta}
\label{sec4}
\begin{center}
\begin{footnotesize}
\begin{verbatim}
   
   Nombre del fichero:
    tiempo_compuesta.py

   Autoras de dicho fichero:
    Nadia Chinea Chinea 
    Tania Gutierrez Gutierrez 
    Melanie Hernandez Alonso

  #!/usr/bin/python
  import random, sys
  from math import *
  import time
  
  e0 = time.time ()
  c0 = time.clock ()
 
  n=4
  def f(x):
    return 1/(sqrt(2*pi))*exp((-x**2)/2)
  
  def simpson1(a,b,x1,x2,x3):
    f1=(((b-a)/n)/3.0)
    f2=(f(a)+(2.0*f(x1))+(4.0*f(x2))+(2.0*f(x3))+f(b))
    return f1*f2 
      
  if __name__ == '__main__':
    if (len(sys.argv)) == 6:
      a=float(sys.argv[1])
      b=float(sys.argv[2])
      x1=float(sys.argv[3])
      x2=float(sys.argv[4])
      x3=float(sys.argv[5])
 
  elapsed_time =  time.time() - e0
  cpu_time = time.clock() - c0
  print ' El tiempo transcurrido con la formula compuesta,es el siguiente: ',elapsed_time
  print ' El tiempo de ejecucion de la CPU es: ',cpu_time
\end{verbatim}
\end{footnotesize}
\end{center}

